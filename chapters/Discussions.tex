
\begin{doublespacing}
The goals of this thesis were to explore the potential of RUS methods for long cylindrical tubes, study the contact pressure between the two cylinders in target, and the usefulness of this technique in assessing target damage due to radiation. The results obtained from RUS are highly dependent on the choice of the geometry. A finite element model was made to predict the frequency, based on geometry, density and elasticity matrix. After predicting the appropriate elastic constants, the RUS spectrum amplitude on different targets with different draw plug was studied. Experimental results from the contact pressure analysis was discussed. Theoretical study was completed to see the effect of radiation on mechanical properties of the material. After carefully studying the radiation effect on material, some recommendations can be made to use this technique properly  inside a hot cell. 

\section{Conclusion}
Experimentally detected frequency based on the quality factor is shown in Figure \ref{freq_Q}. Minimum frequency was detected  around 250 kHz. Theoretically, below 250 kHz some modes existed, but due to the length and weight of the sample all these mode frequencies had undetectable responses. In RUS, a Demarest plot is usually used to measure the experimental Poisson's ratio. However, to generate a complete Demarest plot, it is necessary to experimentally  detect first 20 modes. A time-of-flight method can be handy in this situation to measure Poisson's ratio for an isotropic material. Even though the finite element method  used in this study was highly time consuming, a simple mesh study was done to see the effect of the meshing on the eigen frequencies. A two pinducer system with a different holding mechanism was used to detect the frequencies below 250 kHz. But because of the size and shape of the sample the experiment was done in "surface to surface" manner. To extract all the frequencies below 300 kHz a study should be done to make a sophisticated detection system. To see the effect on the spectrum three different targets were made with three different plug sizes, which produced distinguishable responses.% Theoretical study was not done for this experiment. 

\section{Recommendations}
Based on the result discussed above, a couple of recommendations can be made. The importance of sophisticated measurement system has already been established. Machining of the sample is very important, any non-uniformity will produce discrepancies in the frequency response. Before using RUS to measure radiation effects, some parameters should be considered. Dimensional change can happen after irradiation (Figure \ref{dim_joyo}). Before making any RUS measurement it is necessary to measure the dimension properly. Temperature has profound effect on the elastic constants. So cooling down of the irradiated material, and proper heat treatment might be necessary for the sample. An experiment can be designed to  measure the contact resistance of the assembled targets. A numerical model of the assembled targets might help to understand the behavior of the RUS responses. 

\section{Future Work}
Future work in extending RUS method should involve looking at the development of a quantitative measurement system for contact pressure between two cylinders. Creating a correlation between the thermal contact resistance and amplitude of the target would help to mathematically explain the phenomena. This idea is based on the experimental results from target. To use RUS in radiation environment, inserting the whole setup in the hot cell is an important step. The transducer response in the radition field should be studied. Study of non-uniformity and anisotropy in the field of RUS may help in studying radiation damage with RUS.


\end{doublespacing}