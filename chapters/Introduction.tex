\section{Introduction}
\begin{doublespacing}
%\subsection{Basic}
Resonant Ultrasound Spectroscopy (RUS) is a measurement technique for material properties of an elastic object that uses resonant frequencies of the material. Resonant frequencies are usually obtained through a very simple experiment using two or more ultrasonic transducers. For anisotropic materials, it is capable of extracting all 21 elastic constants. Exciting the object and recording the responses are the two major parts of the experiment. The main advantage of RUS is the simplicity of the experiment. Theoretically, the vibration spectrum of an elastic object contains much information about the object, both microscopic and macroscopic. Thermodynamic properties e.g. entropy and Helmholz free energy can also be measured using RUS \cite{rusbook} \cite{anderson1995equations}. 


Obtaining the elastic constants from the spectrum is usually referred as the "inverse problem" in RUS. The "forward problem" is the solution of the mathematical model which depends on the shape of the object. This means that a numerical approximation is used to predict the resonant spectrum of the object, and this approximation is compared to the measured spectrum. If the measured and calculated spectrum do not agree within the acceptable limits, the parameters used in the calculation are adjusted and the computation is redone until the error reaches to a tolerable limit. This iterative process is the basis of the RUS method. Initially the RUS method was limited to particular types of shapes, with high symmetry such as cylinders, spheres, and rectangular parallelepipeds (RP) \cite{litwiller2000resonant}. But more recently attempts have been made to study arbitrarily shaped objects by Maynard \cite{liu2012measuring}. Study of carbon nanotubes was the first attempt to study the hollow geometry \cite{li2008acoustic}.

\subsection{History}
It was known before the industrial revolution that sound carries information of the vibrating object. Before any scientific development, bell makers around the world knew that different alloys produce different sound, and the shape of the object is important in the quality of the sound. This is an early example how shape and material influence the sound quality. Presence of any kind of crack or imperfection changes the sound quality. Knowledge of the relationship between shape, material, and resonant frequency was used in a variety of fields. Before World War I, British railroad engineers had a similar type of technique. They would tap the wheels of the train and use the sound to detect cracks. This may be the first engineering use of sound to test material performance \cite{migliori1996resonant}. Later attempts were made to understand the resonance of solids mathematically based on the shape and composition. Nobel laureate Lord Rayleigh's contribution in the late eighteenth century was significant. In World War II German engineers utilized a violinist to rub his bow over different turbine blade designs to verify that the resonant frequencies of the blades would be very different from the angular speed of the engine itself to avoid the catastrophic failure that could result from the resonant frequency of the blade \cite{hallion1983designers}. The knowledge of natural resonance is also important in structure design.

Interest in elastic properties goes back to the 17th century when Galileo and other philosophers studied the static equilibrium of bending beams. In 1660 Robert Hooke first developed the law of elasticity. With the help of mathematicians such as Leonhard Euler, Joseph Lagrange, George Green, Simeon-Denis Poisson and others, the mathematical understanding of the elasticity became very strong. Augustus Love summarized all of these theories in his quintessential book, "\emph{Treatise on the Mathematical Theory of Elasticity}", in 1927. The theory of elasticity suggests that the elastic constants of a material could be determined through measurement of the sound velocities in the material. This gave rise to the idea of conventional time-of-flight measurement with ultrasonic pulses.

Natural frequency measurements were used as early as 1935, but these early methods could only find approximate solutions \cite{demarest1971cube}. Gabriel La{\'m}e and Horace Lamb found analytic solutions to the forward problem for a spherical shape with isotropic and noncrystalline materials \cite{love2013treatise}. These investigations were focused mainly on the Earth after a large earthquake. The geophysics community contributed in this particular problem with a view to determining the Earth's interior structure, and to measure accurately the elastic moduli of materials believed to be earth's constituents \cite{maynard2008resonant}. These investigators did not solve the inverse problem for sphere. 

In 1964 Fraiser and LeCrew \cite{fraser1964novel} used the solution of the sphere and inverted graphically, which may be the first RUS measurement. Geophysicists Orson Anderson, Naohiro Soga and Edward Schreiber who worked to improve Fraiser and LeCrew's method, introduced the term Resonant Ultrasound Technique (RST). Anderson and Schreiber used RST to measure spherical lunar samples in 1970. In their paper,  an observation about the composition of the moon was made and some ancient claims about moon being made of cheese were studied. In Erasmus's collection of Latin proverbs, Adagia's(1452) phrase, "his friends believed the moon to be made of Green cheese" was quoted \cite{anderson1970elastic}. Sound speed of different cheeses and lunar rocks were also studied.

Harod Demarest, who was a graduate student of Anderson at Columbia University, first used RST for a cubic shape.  He is the first one to approach the forward problem numerically rather than analytically, based on variational principles. The Rayleigh-Ritz technique and integral equations were used by Demarest  to define  the free vibrations.  He referred this technique as \emph{cube resonance}. Mineo Kumazawa, a post-doctoral researcher at Columbia university was pursuing ultrasonic measurements of minerals during the lunar rocks experiments. He studied Demarest's method and when he returned to Nagoya University, established a laboratory on resonant ultrasound.  His graduate student, Ichiro Ohno worked with him to develop the technique. In 1976 Ohno published a paper with some significant development of Demarest's work \cite{ohno1976free}. In 1988 while working with small piezoelectric film transducers, Migliori and Maynard faced problem with the resonance frequencies and Migliori tracked down the geophysics literature for RUS. Migliori extended the technique and introduced the term \emph{Resonant Ultrasound Spectroscopy}. Even with the limitation of geometry choice, this technique soon proved its usefulness in material science. In the early years the biggest problem was the limitation of computational resources. With the advancement of computing power, RUS is becoming more useful in the field of material research.




\end{doublespacing}



