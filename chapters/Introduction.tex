\section{Introduction}
\begin{doublespacing}
Thermal conductivity is one of the important properties in designing a nuclear fuel, since most of the physical phenomenon are thermally governed. In case of high-performance reactors, fuel goes through high fission density at relatively low temperatures. For this reason research reactor fuels are designed for efficient heat rejection. Usually it is composed of assemblies of thin-plates that uses aluminum alloy cladding. This is called \lq dispersion\rq\ fuels.

The Reduced Enrichment for Research and Test Reactors (RERR) program was initiated in the US in the late 1970s, to develop new fuels. This development of Low-Enriched Uranium (LEU) fuels for high-performence reactors is an important nonproliferatio initiatives~\cite{snelgrove1997development}. One of the major requirements is having higher Uranium density to offset the decrease in enrichment. Different alloys have been selected and tested. The U-Mo alloy has been identified as a high performence fuel due to its hgih Uranium density and low neutron capture cross-section~\cite{ewh2010microstructural,smirnova2013ternary,rest2009analysis,landa2013density}.

\end{doublespacing}



