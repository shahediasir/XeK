%\section{Introduction}
\begin{doublespacing}

 The Reduced Enrichment for Research and Test Reactors (RERR) program was initiated in the US in the late 1970s, to develop new fuels to replace High Enriched Uranium. This development of Low-Enriched Uranium (LEU) fuels for high-performence reactors is an important nonproliferatio initiatives~\cite{snelgrove1997development}. One of the major requirements is having higher Uranium density to offset the decrease in enrichment. Different alloys have been selected and tested. The U-Mo alloy has been identified as a high performence fuel due to its hgih Uranium density and low neutron capture cross-section~\cite{ewh2010microstructural,smirnova2013ternary,rest2009analysis,landa2013density}. The gamma U phase is stabilized by Mo in solid solution and provide acceptable irradiation property. Uranium alloyed with nominally $10\%$ Mo (U-10Mo) is currently being developed as a potential LEU fuel. 


Thermal conductivity is one of the important properties in designing a nuclear fuel, since most of the physical phenomenon are thermally governed. In case of high-performance reactors, fuel goes through high fission density at relatively low temperatures. For this reason research reactor fuels are designed for efficient heat rejection. Usually it is composed of assemblies of thin-plates that uses aluminum alloy cladding. During test operations, it was observed that thermal conductivity decreases with increased burnup. For a fission density of $3.30\times10^{21}$ fissions $cm^{-3}$, at $200^0C$ thermal conductivity decreases about $30\%$, for $4.53\times10^{21}$ fissions $cm^{-3}$ it becomes $45\%$~\cite{burkes2015thermal}. Fission also creates a variety of fission products. These fission products consists of gas bubbles, oxide precipitates, metallic precipitates and solid solution in the fuel matrix~\cite{rondinella2010high}. Radiation damage and fission products in reactor environment results in complex microstructure evolution which restructures the nuclear fuel over the period of time. Microsturctural evolution depends on the radial direction of the fuel~\cite{stehle1988performance,noirot2008detailed, meyer2014irradiation}. For every four fission events one inert gas atom (Xe or Kr) is produced. The dominant type is Xenon (Xe) because of its largest concentration. Xe atoms in U-Mo alloy fuel have a strong tendency to precipitate into small bubble due to their low solubility. The formation and growth of gas bubbles inside irradiated nulcear fuels has techniqual importance since it  influences the microstructur of the fuel material~\cite{kim2011fission}. Recent TEM and SEM images show that fission bubble in U-10Mo distribute itself in intergranular and intragranular format~\cite{miller2015transmission,miller2012advantages, gan2010transmission, gan2012tem}. High fission density microstructure shows randomly distributed micron sized fission-gas bubbles distributed throughout the grain boundary. Intergranular bubble density increases with the increase in burnup. Inside the grain fission gas forms superlattices. Superlattices have been seen before in the ion irradiated matrial~\cite{mazey1986bubble, johnson2006helium, johnson1991image, evans1986solid, johnson1980gas, johnson1980hydrogen, lawson1998temperature, evans1983void, johnson1995gas, ghoniem2001theory}. Typical bubble sizes and spacing in these bubbles are in the range of 2-6.4 nm and 4-12 nm respectively. Usually superlattice has the same crystal structure as the host material, with some exception in some materials. Superlattice in U-10Mo shows a FCC structure in BCC matrix. Ion irradiated bubble superlattice has superlattice constant of 10's of nanometer.

Inclusions and porosity change the thermal and the electrical conductivity of many material. Various models both empirical and analytical, have been proposed to describe this influence. Maxwell~\cite{maxwell1881treatise}, was the first one to derive an expression for effective thermal conductivity with an assumption of idealized distribution of spherical particles in a matrix. Few other empirical formulas~\cite{macewan1967effect,goldsmith1973measurements,devries1989experimental} also exist. The large diversity of shape, porosity and inclusion inside the materials make it impossible to come with one singular equation. Several analytical models describe the influence of porosity and inclusions on the thermal conductivity~\cite{maxwell1881treatise,loeb1954thermal, cunningham1981heat, tzou1991effect, bauer1993general}. These theoritical models are usually valid for pore with regular shape, and the pore arrangment is sufficiently dilute. Numerical methods have also been developed to approximate the effect of porosity and heterogenous microstructure on thermal conductivity. Bakker et. al~\cite{bakker1995determination, bakker1997using} reconstructed irregular shaped bubble from micrograph of irradiated fuel and meshed for Finite Element Analysis (FEM). Yun et. al~\cite{yun2014simulation} performed heat transfer simulation in metallic U-10Zr fuel with distributed spherical pores. Teague et. al.~\cite{teague2014using} studied porosity and precipitates in irradited-mixed oxide fuels using 3D microstructure. Recently Hu et al.~\cite{hu2015assessment} studied the impact of distributed gas bubble on the effective thermal conductivity of U-Mo metallic fuels using phase field modelling. Microstructure of irradiated nuclear fuels are very complicated. The size difference between intra and intergranular bubble is very large. The shape is also highly dissimmilar. The intragranular gas bubbles have spherical shape, while intergranular gas have no regular shape. In this work, thermal conductivity  both inter and intragranular gas bubbles are used to assess the effective thermal conductivity of U-10Mo. The impact of pressure on fission gas's (Xenon) thermal conductivity is also studied, in order to choose a suitable combination of temperature and pressure. 



\end{doublespacing}



